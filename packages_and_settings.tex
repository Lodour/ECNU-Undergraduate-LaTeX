\usepackage{fancybox,fancyvrb,shortvrb} %也许需要的宏包
\usepackage[heading]{ctex} %用来提供中文支持
\usepackage{amsmath} %
\usepackage{amssymb} %数学符号,定理等环境相关宏包
\usepackage{amsthm}  %
\usepackage{graphicx} %插入图片所需宏包
\usepackage{adjustbox}  %也许需要的宏包
\usepackage{xspace} %提供一些好用的空格命令
\usepackage{tikz-cd} %画交换图需要的宏包
\usepackage{url} %更好的超链接显示
\usepackage{array} %表格相关的宏包
\usepackage{booktabs} %表格相关的宏包
\usepackage{caption} %实现图片的多行说明
\usepackage{float} %图片与表格的更好排版
\usepackage{ulem} %更好的下划线

\usepackage[top=2.5cm, bottom=2.0cm, left=3.0cm, right=2.0cm]{geometry} %设置页边距

\usepackage{fontspec}                   %设置字体需要的宏包
\setmainfont{Times New Roman}           %设置西文字体为Times New Roman
\setCJKmainfont{SimSun}                 %设置中文字体为宋体
\renewcommand{\normalsize}{\zihao{-4}}  %设置正文字号为小四

\linespread{1.5} %1.5倍行距

\showboxdepth=5
\showboxbreadth=5 

%设置各级系统的编号格式
\setcounter{secnumdepth}{5}                                                                                     
\ctexset { section = { name={,、},number={\chinese{section}},format={\centering \heiti \zihao {-4}} } }
\ctexset { subsection = { name={(,)},number={\chinese{subsection}},format={\centering \heiti \zihao {-4}} } }
\ctexset { subsubsection = { name={,.},number={\arabic{subsubsection}},format={\heiti \zihao {-4}} } }
\ctexset { paragraph = { name={(,)},number={\arabic{paragraph}},format={\heiti \zihao {-4}} } }
\ctexset { subparagraph = { name={,)},number={\arabic{subparagraph}},format={\heiti \zihao {-4}} } }

\usepackage[bottom,perpage]{footmisc}               %脚注,显示在每页底部,编号按页重置
\renewcommand*{\footnotelayout}{\zihao{-5}\songti}  %设置脚注为小五号宋体
\renewcommand{\thefootnote}{[\arabic{footnote}]}    %设置脚注标记为  [编号]

%设置页眉页脚
\usepackage{fancyhdr}
\lhead{华东师范大学学士学位论文}
\chead{}
\rhead{\TitleCHS}
\lfoot{}
\cfoot{}
\rfoot{\thepage}

\usepackage{xcolor} %彩色的文字

\usepackage[hidelinks]{hyperref} %各种超链接必备
\usepackage{cleveref} %交叉引用

%设置尾注
\usepackage{endnotes}
\renewcommand{\enotesize}{\zihao{-5}}
\renewcommand{\notesname}{\heiti \zihao {-4} 尾注}
\renewcommand\enoteformat{
	\raggedright
	\leftskip=1.8em
	\makebox[0pt][r]{\theenmark. \rule{0pt}{\dimexpr\ht\strutbox+\baselineskip}}
}
\renewcommand\makeenmark{\textsuperscript{[尾注\theenmark]}}
\usepackage{footnotebackref}  

%定义各种常用环境
\newtheorem{theorem}{\heiti 定理}[section]
\newtheorem*{theorem*}{\heiti 定理}
\newtheorem{lemma}[theorem]{\heiti 引理}
\newtheorem*{lemma*}{\heiti 引理}
\newtheorem{corollary}[theorem]{\heiti 推论}
\newtheorem*{corollary*}{\heiti 推论}
\newtheorem{definition}[theorem]{\heiti 定义}
\newtheorem*{definition*}{\heiti 定义}
\newtheorem{conjecture}[theorem]{\heiti 猜想}
\newtheorem*{conjecture*}{\heiti 猜想}
\newtheorem{problem}[theorem]{\heiti 问题}
\newtheorem*{problem*}{\heiti 问题}
\newtheorem{proposition}[theorem]{\heiti 命题}
\newtheorem*{proposition*}{\heiti 命题}
\newtheorem{remark}[theorem]{\heiti 注记}
\newtheorem*{remark*}{\heiti 注记}
\newtheorem{example}[theorem]{\heiti 例}
\newtheorem*{example*}{\heiti 例}
\newenvironment{solution}
{
  \renewcommand\qedsymbol{$\blacksquare$}
	\begin{proof}[\heiti \bf 解]
}
{
	\end{proof}
}
\renewcommand*{\proofname}{\heiti \bf 证明}
	
%设置各种常用环境的交叉引用格式
\crefformat{theorem}{#2\bf{\heiti 定理} #1#3}
\crefformat{lemma}{#2\bf{\heiti 引理} #1#3}
\crefformat{corollary}{#2\bf{\heiti 推论} #1#3}
\crefformat{definition}{#2\bf{\heiti 定义} #1#3}
\crefformat{conjecture}{#2\bf{\heiti 猜想} #1#3}
\crefformat{problem}{#2\bf{\heiti 问题} #1#3}
\crefformat{proposition}{#2\bf{\heiti 命题} #1#3}
\crefformat{remark}{#2\bf{\heiti 注记} #1#3}
\crefformat{example}{#2\bf{\heiti 例} #1#3}
	
%允许公式跨页显示
\allowdisplaybreaks
	
%使用biblatex管理文献,输出格式使用gb7714-2015标准,后端为biber
\usepackage[bibstyle=gb7714-2015,citestyle=gb7714-2015,hyperref=true,backend=biber,sorting=none]{biblatex}
	
%提供了附录支持并显示在目录中
\usepackage[titletoc,title]{appendix} 
\renewcommand{\appendixtocname}{附录}
	
%重定义生成附录的命令,使得每个附录都单独成页
\newcommand{\apdx}[1] {
	\clearpage
	\section{#1}}
	
%生成附录,请勿改动
\newcommand{\makeapdx}{
	\clearpage
	\begin{appendices}
		\renewcommand{\thesection}{\chinese{section}、}
		\input{./ending/Appendix.tex}
	\end{appendices}
}
	
%生成感谢,请勿改动
\newcommand{\makeacknowledgement}{
	\clearpage
	\input{./ending/acknowledgement.tex}
}
	
%For Algorithm
\usepackage{algorithm}
\usepackage{algorithmicx}
\usepackage{algpseudocode}
\floatname{algorithm}{算法}
\renewcommand{\algorithmicrequire}{\textbf{输入:}}
\renewcommand{\algorithmicensure}{\textbf{输出:}}
	
%使表格中的脚注也能够正常显示
%\usepackage{footnote}
%\makesavenoteenv{table}
	
%可能会需要在用自然语言描述算法步骤时使用的宏包
\usepackage{enumitem}
	
%表格单元格内换行
\newcommand{\tabincell}[2]{\begin{tabular}{@{}#1@{}}#2\end{tabular}}
	
%设置目录字体
\usepackage{tocloft}
\renewcommand{\contentsname}{\hfill \bf \heiti \zihao{-4} 目\hspace*{2em}录 \hfill}   
\renewcommand{\cftaftertoctitle}{\hfill}
\renewcommand{\cfttoctitlefont}{\heiti}
\renewcommand{\cftsubsubsecfont}{\heiti}
\renewcommand{\cftsubsecfont}{\heiti}
\renewcommand{\cftsecfont}{\bf \heiti}
	
	
\usepackage{soul} %又一个下划线包
\usepackage{setspace} %灵活的行距定义(用于封面)
