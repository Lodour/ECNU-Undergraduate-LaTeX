\usepackage{fancybox,fancyvrb,shortvrb} %也许需要的宏包
\usepackage[heading]{ctex} %用来提供中文支持
\usepackage{amsmath} %
\usepackage{amssymb} %数学符号,定理等环境相关宏包
\usepackage{amsthm}  %
\usepackage{graphicx} %插入图片所需宏包
\usepackage{adjustbox}  %也许需要的宏包
\usepackage{xspace} %提供一些好用的空格命令
\usepackage{tikz-cd} %画交换图需要的宏包
\usepackage{url} %更好的超链接显示
\usepackage{array} %表格相关的宏包
\usepackage{booktabs} %表格相关的宏包
\usepackage{caption} %实现图片的多行说明
\usepackage{float} %图片与表格的更好排版



\usepackage{ulem} %更好的下划线

\usepackage[ top=2.5cm, bottom=2.0cm, left=3.0cm, right=2.0cm]{geometry} %设置页边距

\usepackage{fontspec}                   %设置字体需要的宏包
\setmainfont{Times New Roman}           %设置西文字体为Times New Roman
\setCJKmainfont{SimSun}                 %设置中文字体为宋体
\renewcommand{\normalsize}{\zihao{-4}}  %设置正文字号为小四

\linespread{1.5} %1.5倍行距

\showboxdepth=5
\showboxbreadth=5 

\setcounter{secnumdepth}{5}                                                                                     %
\ctexset { section = { name={,、},number={\chinese{section}},format={\centering \heiti \zihao {-4}} } }         %
\ctexset { subsection = { name={(,)},number={\chinese{subsection}},format={\centering \heiti \zihao {-4}} } } %设置各级系统的编号格式
\ctexset { subsubsection = { name={,.},number={\arabic{subsubsection}},format={\heiti \zihao {-4}} } }          %
\ctexset { paragraph = { name={(,)},number={\arabic{paragraph}},format={\heiti \zihao {-4}} } }               %
\ctexset { subparagraph = { name={,)},number={\arabic{subparagraph}},format={\heiti \zihao {-4}} } }           %

\usepackage[bottom,perpage]{footmisc}               %脚注,显示在每页底部,编号按页重置
\renewcommand*{\footnotelayout}{\zihao{-5}\songti}  %设置脚注为小五号宋体
\renewcommand{\thefootnote}{[\arabic{footnote}]}    %设置脚注标记为  [编号]
                      %脚注的反向超链接


\usepackage{fancyhdr}               %
\renewcommand{\headrulewidth}{0pt}  %
\lhead{}                            %
\chead{}                            %将页眉页脚设置为:仅在右下角显示页码
\rhead{}                            %
\lfoot{}                            %
\cfoot{}                            %
\rfoot{\thepage}                    %

\usepackage{xcolor} %彩色的文字

\usepackage[hidelinks]{hyperref} %各种超链接必备

\usepackage{endnotes}                                                           %
\renewcommand{\enotesize}{\zihao{-5}}                                           %
\renewcommand{\notesname}{\heiti \zihao {-4} 尾注}                              %
\renewcommand\enoteformat{                                                      %
  \raggedright                                                                  %尾注的相关设置
  \leftskip=1.8em                                                               %
  \makebox[0pt][r]{\theenmark. \rule{0pt}{\dimexpr\ht\strutbox+\baselineskip}}  %
}                                                                               %
\renewcommand\makeenmark{\textsuperscript{[尾注\theenmark]}}                    %
\usepackage{footnotebackref}  


\newtheorem{theorem}{\heiti 定理}[section]      %
\newtheorem*{theorem*}{\heiti 定理}             %
\newtheorem{lemma}[theorem]{\heiti 引理}        %
\newtheorem*{lemma*}{\heiti 引理}               %
\newtheorem{corollary}[theorem]{\heiti 推论}    %
\newtheorem*{corollary*}{\heiti 推论}           %
\newtheorem{definition}[theorem]{\heiti 定义}   %
\newtheorem*{definition*}{\heiti 定义}          %
\newtheorem{conjecture}[theorem]{\heiti 猜想}   %将各种常用环境设置为中文
\newtheorem*{conjecture*}{\heiti 猜想}          %
\newtheorem{problem}[theorem]{\heiti 问题}      %
\newtheorem*{problem*}{\heiti 问题}             %
\newenvironment{solution}                       %
  {\renewcommand\qedsymbol{$\blacksquare$}      %
  \begin{proof}[\heiti \bf 解]}                 %
  {\end{proof}}                                 %
\renewcommand*{\proofname}{\heiti \bf 证明}     %

\allowdisplaybreaks %允许公式跨页显示


\usepackage[bibstyle=gb7714-2015,citestyle=gb7714-2015,hyperref=true,backend=biber,sorting=none]{biblatex} %使用biblatex管理文献,输出格式使用gb7714-2015标准,后端为biber

\usepackage[titletoc,title]{appendix} %提供了附录支持并显示在目录中
\renewcommand{\appendixtocname}{附录} %

\newcommand{\apdx}[1] { %
\clearpage              %重定义生成附录的命令,使得每个附录都单独成页
\section{#1}}           %

%For Algorithm
\usepackage{algorithm}
\usepackage{algorithmicx}
\usepackage{algpseudocode}
\floatname{algorithm}{算法}
\renewcommand{\algorithmicrequire}{\textbf{输入:}}
\renewcommand{\algorithmicensure}{\textbf{输出:}}

%使表格中的脚注也能够正常显示
\usepackage{footnote}
\makesavenoteenv{table}

%可能会需要在用自然语言描述算法步骤时使用的宏包
\usepackage{enumitem}

%表格单元格内换行
\newcommand{\tabincell}[2]{\begin{tabular}{@{}#1@{}}#2\end{tabular}}
